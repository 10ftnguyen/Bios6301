\documentclass[]{article}
\usepackage[T1]{fontenc}
\usepackage{lmodern}
\usepackage{amssymb,amsmath}
\usepackage{ifxetex,ifluatex}
\usepackage{fixltx2e} % provides \textsubscript
% use microtype if available
\IfFileExists{microtype.sty}{\usepackage{microtype}}{}
\ifnum 0\ifxetex 1\fi\ifluatex 1\fi=0 % if pdftex
  \usepackage[utf8]{inputenc}
\else % if luatex or xelatex
  \usepackage{fontspec}
  \ifxetex
    \usepackage{xltxtra,xunicode}
  \fi
  \defaultfontfeatures{Mapping=tex-text,Scale=MatchLowercase}
  \newcommand{\euro}{€}
\fi
% Redefine labelwidth for lists; otherwise, the enumerate package will cause
% markers to extend beyond the left margin.
\makeatletter\AtBeginDocument{%
  \renewcommand{\@listi}
    {\setlength{\labelwidth}{4em}}
}\makeatother
\usepackage{enumerate}
\ifxetex
  \usepackage[setpagesize=false, % page size defined by xetex
              unicode=false, % unicode breaks when used with xetex
              xetex]{hyperref}
\else
  \usepackage[unicode=true]{hyperref}
\fi
\hypersetup{breaklinks=true,
            bookmarks=true,
            pdfauthor={},
            pdftitle={},
            colorlinks=true,
            urlcolor=blue,
            linkcolor=magenta,
            pdfborder={0 0 0}}
\setlength{\parindent}{0pt}
\setlength{\parskip}{6pt plus 2pt minus 1pt}
\setlength{\emergencystretch}{3em}  % prevent overfull lines
\setcounter{secnumdepth}{0}

\author{}
\date{}

\begin{document}

\section{Bios 301: Assignment 2}

\emph{Due Thursday, 31 October 👻 2013, 12:00 PM}

50 points total.

Submit a single knitr (either \texttt{.rnw} or \texttt{.rmd}) file,
along with a valid PDF output file. Inside the file, clearly indicate
which parts of your responses go with which problems (you may use the
original homework document as a template). Raw R code/output or word
processor files are not acceptable.

\subsubsection{Question 1}

\textbf{20 points}

A problem with the Newton-Raphson algorithm is that it needs the
derivative $f′$. If the derivative is hard to compute or does not exist,
then we can use the \emph{secant method}, which only requires that the
function $f$ is continuous.

Like the Newton-Raphson method, the \textbf{secant method} is based on a
linear approximation to the function $f$. Suppose that $f$ has a root at
$a$. For this method we assume that we have \emph{two} current guesses,
$x_0$ and $x_1$, for the value of $a$. We will think of $x_0$ as an
older guess and we want to replace the pair $x_0$, $x_1$ by the pair
$x_1$, $x_2$, where $x_2$ is a new guess.

To find a good new guess x2 we first draw the straight line from
$(x_0,f(x_0))$ to $(x_1,f(x_1))$, which is called a secant of the curve
$y = f(x)$. Like the tangent, the secant is a linear approximation of
the behavior of $y = f(x)$, in the region of the points $x_0$ and $x_1$.
As the new guess we will use the x-coordinate $x_2$ of the point at
which the secant crosses the x-axis.

The general form of the recurrence equation for the secant method is:

\[x_{i+1} = x_i - f(x_i)\frac{x_i - x_{i-1}}{f(x_i) - f(x_{i-1})}\]

Notice that we no longer need to know $f′$ but in return we have to
provide \emph{two} initial points, $x_0$ and $x_1$.

\textbf{Write a function that implements the secant algorithm.} Validate
your program by finding the root of the function $f(x) = \cos(x) - x$.
Compare its performance with that of the either the Newton-Raphson or
the Fixed-point method -- which is faster, and by how much?

\subsubsection{Question 2}

\textbf{15 points}

Import the HAART dataset (\texttt{haart.csv}) from the GitHub repository
into R, and perform the following manipulations:

\begin{enumerate}[1.]
\item
  Convert date columns into a usable (for analysis) format.
\item
  Create an indicator variable (one which takes the values 0 or 1 only)
  to represent death within 1 year of the initial visit.
\item
  Use the \texttt{init.date}, \texttt{last visit} and
  \texttt{death.date} to calculate a followup time, which is the
  difference between the first and either the last visit or a death
  event (whichever comes first). If these times are longer than 1 year,
  censor them.
\item
  Create another indicator variable representing loss to followup; that
  is, if their status 1 year after the first visit was unknown.
\item
  Recall our work in class, which separated the \texttt{init.reg} field
  into a set of indicator variables, one for each unique drug. Create
  these fields and append them to the database as new columns.
\item
  The dataset \texttt{haart2.csv} contains a few additional observations
  for the same study. Import these and append them to your master
  dataset (if you were smart about how you coded the previous steps,
  cleaning the additional observations should be easy!).
\end{enumerate}

\subsubsection{Question 3}

\textbf{15 points}

The game of craps is played as follows. First, you roll two six-sided
dice; let x be the sum of the dice on the first roll. If x = 7 or 11 you
win, otherwise you keep rolling until either you get x again, in which
case you also win, or until you get a 7 or 11, in which case you lose.

Write a program to simulate a game of craps. You can use the following
snippet of code to simulate the roll of two (fair) dice:

\begin{verbatim}
x <- sum(ceiling(6*runif(2)))
\end{verbatim}

The instructor should be able to easily import and run your program
(function), and obtain output that clearly shows how the game
progressed.

\end{document}
