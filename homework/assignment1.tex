\documentclass[]{article}
\usepackage[T1]{fontenc}
\usepackage{lmodern}
\usepackage{amssymb,amsmath}
\usepackage{ifxetex,ifluatex}
\usepackage{fixltx2e} % provides \textsubscript
% use microtype if available
\IfFileExists{microtype.sty}{\usepackage{microtype}}{}
\ifnum 0\ifxetex 1\fi\ifluatex 1\fi=0 % if pdftex
  \usepackage[utf8]{inputenc}
\else % if luatex or xelatex
  \usepackage{fontspec}
  \ifxetex
    \usepackage{xltxtra,xunicode}
  \fi
  \defaultfontfeatures{Mapping=tex-text,Scale=MatchLowercase}
  \newcommand{\euro}{€}
\fi
% Redefine labelwidth for lists; otherwise, the enumerate package will cause
% markers to extend beyond the left margin.
\makeatletter\AtBeginDocument{%
  \renewcommand{\@listi}
    {\setlength{\labelwidth}{4em}}
}\makeatother
\usepackage{enumerate}
\ifxetex
  \usepackage[setpagesize=false, % page size defined by xetex
              unicode=false, % unicode breaks when used with xetex
              xetex]{hyperref}
\else
  \usepackage[unicode=true]{hyperref}
\fi
\hypersetup{breaklinks=true,
            bookmarks=true,
            pdfauthor={},
            pdftitle={},
            colorlinks=true,
            urlcolor=blue,
            linkcolor=magenta,
            pdfborder={0 0 0}}
\setlength{\parindent}{0pt}
\setlength{\parskip}{6pt plus 2pt minus 1pt}
\setlength{\emergencystretch}{3em}  % prevent overfull lines
\setcounter{secnumdepth}{0}

\author{}
\date{}

\begin{document}

\section{Bios 301: Assignment 1}

\emph{Due Tuesday, 8 October, 12:00 PM}

50 points total.

Submit a single knitr (either \texttt{.rnw} or \texttt{.rmd}) file,
along with a valid PDF output file. Inside the file, clearly indicate
which parts of your responses go with which problems (you may use the
original homework document as a template). Raw R code/output or word
processor files are not acceptable.

\begin{enumerate}[1.]
\item
  \textbf{Working with data} In the \texttt{datasets} folder on the
  course GitHub repo, you will find a file called \texttt{cancer.csv},
  which is a dataset in comma-separated values (csv) format. This is a
  large cancer incidence dataset that summarizes the incidence of
  different cancers for various subgroups. (18 points)

  \begin{enumerate}[1.]
  \item
    Load the data set into R and make it a data frame called
    \texttt{cancer.df}. (2 points)
  \item
    Determine the number of rows and columns in the data frame. (2)
  \item
    Extract the names of the columns in \texttt{cancer.df}. (2)
  \item
    Report the value of the 3000th row in column 6. (2)
  \item
    Report the contents of the 172nd row. (2)
  \item
    Create a new column that is the incidence \emph{rate} (per 100,000)
    for each row.(3)
  \item
    How many subgroups (rows) have a zero incidence rate? (2)
  \item
    Find the subgroup with the highest incidence rate.(3)
  \end{enumerate}
\item
  \textbf{Data types} (14 points)

  \begin{enumerate}[1.]
  \item
    Create the following vector:
    \texttt{x \textless{}- c("5","12","7")}. Which of the following
    commands will produce an error message? For each command, Either
    explain why they should be errors, or explain the non-erroneous
    result. (6 points)

\begin{verbatim}
max(x)
sort(x)
sum(x)
\end{verbatim}
  \item
    For the next two commands, either explain their results, or why they
    should produce errors. (4 points)

\begin{verbatim}
y <- c("5",7,12)
y[2] + y[3]
\end{verbatim}
  \item
    For the next two commands, either explain their results, or why they
    should produce errors. (4 points)

\begin{verbatim}
z <- data.frame(z1="5",z2=7,z3=12)
z[1,2] + z[1,3]
\end{verbatim}
  \end{enumerate}
\item
  \textbf{Data structures} Give R expressions that return the following
  matrices and vectors (\emph{i.e.} do not construct them manually). (3
  points each, 12 total)

  \begin{enumerate}[1.]
  \item
    $(1,2,3,4,5,6,7,8,7,6,5,4,3,2,1)$
  \item
    $(1,2,2,3,3,3,4,4,4,4,5,5,5,5,5)$
  \item
    $\left({ \begin{array}{ccc} 0 & 1 & 1 \\
    0 & 0 & 1 \\
    1 & 1 & 0
    \end{array} }\right)$
  \item
    $\left({ \begin{array}{ccc} 0 & 2 & 3 \\
    0 & 5 & 0 \\
    7 & 0 & 0
    \end{array} }\right)$
  \end{enumerate}
\item
  \textbf{Basic programming} Let
  $h(x,n)=1+x+x^2+\ldots+x^n = \sum_{i=0}^n x_i$. Write an R program to
  calculate $h(x,n)$ using a \texttt{for} loop. (6 points)
\end{enumerate}

\end{document}
